\begin{abstract}
 This paper analyzes the Signal Protocol, the de facto standard for end-to-end encrypted messaging. We detail the cryptographic mechanisms of the X3DH key agreement and the Double Ratchet algorithm, which provide Forward Secrecy and Post-Compromise Security. The study examines the WhatsApp implementation, highlighting the paradox between its secure content encryption and the extensive metadata collection authorized by its 2021 privacy policy. Finally, we discuss the Matrix protocol and its Olm/Megolm libraries, which adapt Signal's primitives to balance security and scalability in federated environments.
\end{abstract}
     
\begin{resumo} 
 Este artigo analisa o Protocolo Signal, o padrão de fato para mensagens criptografadas ponta-a-ponta. Detalhamos os mecanismos criptográficos do acordo de chaves X3DH e do algoritmo \textit{Double Ratchet}, que garantem Sigilo Perfeito Encaminhado e Segurança Pós-Comprometimento. O estudo examina a implementação do WhatsApp, destacando o paradoxo entre a criptografia segura do conteúdo e a coleta massiva de metadados autorizada por sua política de privacidade de 2021. Por fim, discutimos o protocolo Matrix e suas bibliotecas Olm/Megolm, que adaptam as primitivas do Signal para equilibrar segurança e escalabilidade em ambientes federados.
\end{resumo}