
\begin{abstract}
\textcolor{red}{[TODOS][REVISAR para garantir consistencia e coesao]}
 This paper analyzes the Signal Protocol, the de facto standard for secure messaging. We explore its evolution from the "Axolotl" ratchet, named after the self-healing salamander, to its current state. Technical concepts such as the Double Ratchet Algorithm and X3DH are detailed. We present case studies on WhatsApp, discussing its closed-source implementation allowed by specific commercial agreements and the controversy surrounding its 2021 privacy policy update regarding metadata collection. Finally, we examine the Matrix protocol and its Olm/Megolm libraries, which adapt Signal's concepts for federated environments and high-performance group chats.
\end{abstract}
     
\begin{resumo} 
\textcolor{red}{[TODOS][REVISAR para garantir consistencia e coesao]}
 Este artigo analisa o Protocolo Signal, o padrão de fato para mensagens seguras. Exploramos sua evolução desde o algoritmo "Axolotl", nomeado em referência à salamandra regenerativa, até seu estado atual. Conceitos técnicos como o algoritmo Double Ratchet e X3DH são detalhados. Apresentamos estudos de caso sobre o WhatsApp, discutindo sua implementação de código fechado permitida por acordos comerciais específicos e a controvérsia em torno da atualização de sua política de privacidade em 2021 referente à coleta de metadados. Por fim, examinamos o protocolo Matrix e suas bibliotecas Olm/Megolm, que adaptam os conceitos do Signal para ambientes federados e grupos de alta performance.
\end{resumo}
