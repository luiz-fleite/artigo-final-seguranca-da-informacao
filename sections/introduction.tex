
\section{Introdução}
\textcolor{red}{[MAX]}
\textcolor{red}{[melhorar contextualização historica para o surgimento da criptografia ponta-aponta]}
A segurança em comunicações digitais evoluiu de modelos baseados em criptografia de transporte (como TLS/SSL) para a Criptografia Ponta-a-Ponta (E2EE), onde nem mesmo o provedor do serviço tem acesso ao conteúdo. O protagonista dessa mudança é o Protocolo Signal.

\subsection{História e o Nome Axolotl}
\textcolor{red}{[MAX]}
\textcolor{red}{[]}
O protocolo foi desenvolvido pela \textit{Open Whisper Systems}, liderada por Moxie Marlinspike e Trevor Perrin. Inicialmente, o mecanismo central de criptografia não se chamava "Signal", mas sim \textbf{Axolotl}.

O nome foi escolhido em homenagem ao axolote (\textit{Ambystoma mexicanum}), uma salamandra aquática conhecida por sua impressionante capacidade de auto-regeneração. Essa analogia biológica referia-se à propriedade de "auto-cura" (\textit{self-healing}) do protocolo: se uma chave de sessão for comprometida por um atacante, o algoritmo rotaciona as chaves automaticamente na próxima mensagem, "curando" a segurança da conversa e impedindo que o atacante decifre mensagens futuras \cite{perrin2016double}.

Em 2016, para simplificar a nomenclatura e evitar confusões de marcas registradas, o nome foi oficialmente alterado para \textbf{Signal Protocol}, unificando a marca do aplicativo e da tecnologia subjacente.