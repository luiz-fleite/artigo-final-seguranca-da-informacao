
\section{Conclusão}
\textcolor{red}{[TODOS][]}
A evolução do algoritmo Axolotl para o onipresente Protocolo Signal redefiniu a segurança na internet. A tecnologia provou ser robusta contra adversários estatais e corporativos no que tange à confidencialidade do conteúdo.

Entretanto, a análise das implementações do WhatsApp e Matrix revela que o protocolo é apenas uma peça do quebra-cabeça. No caso do WhatsApp, a segurança criptográfica convive com a vigilância de metadados sancionada pelos termos de uso. No caso do Matrix, vemos a adaptação do protocolo (Megolm) para garantir soberania de dados através da federação. Conclui-se que a "privacidade" é um conceito mais amplo que a "criptografia", dependendo tanto da matemática dos algoritmos quanto da ética das implementações.
