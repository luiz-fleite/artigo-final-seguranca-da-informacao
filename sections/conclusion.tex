\section{Conclusão}

A análise realizada confirma que o Protocolo Signal se consolidou como o padrão \textit{de facto} para comunicação segura, fundamentado nas propriedades de Sigilo Perfeito Encaminhado e Segurança Pós-Comprometimento garantidas pelos algoritmos X3DH e \textit{Double Ratchet}. A verificação formal dessas primitivas assegura que o conteúdo das mensagens permaneça inacessível a terceiros, mesmo diante de adversários sofisticados.

Contudo, a investigação das implementações revela desafios que vão além da criptografia. O caso do WhatsApp evidencia a dicotomia entre a proteção do conteúdo e a exposição de metadados, demonstrando que a cifragem ponta-a-ponta, isoladamente, não impede a perfilagem comportamental dos usuários pela infraestrutura central. Por outro lado, o protocolo Matrix apresenta uma alternativa viável para ambientes federados, onde a adaptação via Megolm sacrifica a auto-cura imediata em favor da escalabilidade necessária para grandes grupos.

Conclui-se que a segurança na mensageria moderna transcende a robustez matemática dos algoritmos, dependendo criticamente de decisões arquiteturais que minimizem a coleta de metadados e respeitem a soberania dos dados dos usuários.