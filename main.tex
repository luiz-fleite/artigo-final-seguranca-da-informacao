\documentclass[12pt]{article}

\usepackage{styles/sbc-template}

\usepackage[T1]{fontenc}
\usepackage{mathptmx}
\usepackage{graphicx,url}
\usepackage[utf8]{inputenc}
\usepackage[brazil]{babel}

\usepackage[hidelinks]{hyperref}

% Define o caminho das imagens
\graphicspath{{images/}}

\sloppy

\title{Análise da Criptografia Ponta-a-Ponta: Do Axolotl às Implementações Massivas no WhatsApp e Matrix}

\author{Luiz Antônio Lima de Freitas Leite\inst{1}, Max José Lobato Pantoja Junior\inst{1}, \\Wesley Pontes Barbosa\inst{1}}

\address{Instituto de Ciências Exatas e Naturais (ICEN) -- Universidade Federal Pará\\ Belém, PA -- Brasil
\email{\{luiz.freitas.leite,max.junior,wesley.pontes.barbosa\}@icen.ufpa.br}
}

\begin{document} 

\maketitle



% RESUMO - TODOS 

\begin{abstract}
\textcolor{red}{[TODOS][REVISAR para garantir consistencia e coesao]}
 This paper analyzes the Signal Protocol, the de facto standard for secure messaging. We explore its evolution from the "Axolotl" ratchet, named after the self-healing salamander, to its current state. Technical concepts such as the Double Ratchet Algorithm and X3DH are detailed. We present case studies on WhatsApp, discussing its closed-source implementation allowed by specific commercial agreements and the controversy surrounding its 2021 privacy policy update regarding metadata collection. Finally, we examine the Matrix protocol and its Olm/Megolm libraries, which adapt Signal's concepts for federated environments and high-performance group chats.
\end{abstract}
     
\begin{resumo} 
\textcolor{red}{[TODOS][REVISAR para garantir consistencia e coesao]}
 Este artigo analisa o Protocolo Signal, o padrão de fato para mensagens seguras. Exploramos sua evolução desde o algoritmo "Axolotl", nomeado em referência à salamandra regenerativa, até seu estado atual. Conceitos técnicos como o algoritmo Double Ratchet e X3DH são detalhados. Apresentamos estudos de caso sobre o WhatsApp, discutindo sua implementação de código fechado permitida por acordos comerciais específicos e a controvérsia em torno da atualização de sua política de privacidade em 2021 referente à coleta de metadados. Por fim, examinamos o protocolo Matrix e suas bibliotecas Olm/Megolm, que adaptam os conceitos do Signal para ambientes federados e grupos de alta performance.
\end{resumo}


% SEÇÃO 1: INTRODUÇÃO - MAX 

\section{Introdução}
\textcolor{red}{[MAX]}
\textcolor{red}{[melhorar contextualização historica para o surgimento da criptografia ponta-aponta]}
A segurança em comunicações digitais evoluiu de modelos baseados em criptografia de transporte (como TLS/SSL) para a Criptografia Ponta-a-Ponta (E2EE), onde nem mesmo o provedor do serviço tem acesso ao conteúdo. O protagonista dessa mudança é o Protocolo Signal.

\subsection{História e o Nome Axolotl}
\textcolor{red}{[MAX]}
\textcolor{red}{[]}
O protocolo foi desenvolvido pela \textit{Open Whisper Systems}, liderada por Moxie Marlinspike e Trevor Perrin. Inicialmente, o mecanismo central de criptografia não se chamava "Signal", mas sim \textbf{Axolotl}.

O nome foi escolhido em homenagem ao axolote (\textit{Ambystoma mexicanum}), uma salamandra aquática conhecida por sua impressionante capacidade de auto-regeneração. Essa analogia biológica referia-se à propriedade de "auto-cura" (\textit{self-healing}) do protocolo: se uma chave de sessão for comprometida por um atacante, o algoritmo rotaciona as chaves automaticamente na próxima mensagem, "curando" a segurança da conversa e impedindo que o atacante decifre mensagens futuras \cite{perrin2016double}.

Em 2016, para simplificar a nomenclatura e evitar confusões de marcas registradas, o nome foi oficialmente alterado para \textbf{Signal Protocol}, unificando a marca do aplicativo e da tecnologia subjacente.

% SEÇÃO 2: FUNDAMENTAÇÃO TEÓRICA - MAX 

\section{Conceitos Básicos}
\textcolor{red}{[MAX]}
\textcolor{red}{[foca em explicar todas as propriedades relevantes pro protocolo signal, ]}
Para compreender o funcionamento do Protocolo Signal, é necessário definir certas propriedades criptográficas fundamentais que ele visa garantir:

\begin{itemize}
    \item \textbf{Sigilo Perfeito Encaminhado (Forward Secrecy):} Garante que, se a chave privada de um usuário for roubada hoje, as mensagens trocadas no passado permaneçam seguras. Isso é obtido através da geração de chaves de sessão efêmeras que são descartadas após o uso.
    \item \textbf{Segurança Pós-Comprometimento (Post-Compromise Security):} Também conhecida como "Break-in Recovery". Refere-se à capacidade do protocolo de restabelecer a segurança após um comprometimento temporário das chaves, através da atualização contínua do material criptográfico.
    \item \textbf{Diffie-Hellman (DH):} Um método que permite a duas partes, que não têm conhecimento prévio uma da outra, estabelecerem conjuntamente uma chave secreta compartilhada em um canal inseguro.
\end{itemize}


% SEÇÃO 3: O PROTOCOLO SIGNAL - LA 
\section{O Protocolo Signal}
\label{sec:signal_protocol}

O Protocolo Signal não é um algoritmo monolítico, mas uma orquestração sofisticada de primitivas criptográficas projetadas para garantir confidencialidade, integridade e autenticidade em ambientes de comunicação assíncrona. Diferente de predecessores como o OTR (\textit{Off-the-Record}), o Signal foi desenhado especificamente para dispositivos móveis, lidando com conectividade intermitente e sessões de longa duração.

Uma das bases da confiança no protocolo reside em sua transparência: suas especificações são públicas e a implementação de referência é mantida como software de código aberto, permitindo auditoria constante pela comunidade acadêmica e de segurança \cite{signal_github}. Além disso, suas propriedades de segurança foram formalmente verificadas, comprovando matematicamente suas garantias de sigilo e autenticação \cite{cohn2017formal}.

O funcionamento do protocolo pode ser dividido em três fases críticas: a publicação de chaves (pré-chaves), o estabelecimento de sessão (X3DH) e a renovação contínua de chaves (Double Ratchet).

\subsection{Infraestrutura de Chaves (Pre-Keys)}
Para permitir que Alice envie uma mensagem segura para Bob mesmo que ele esteja \textit{offline}, o protocolo utiliza um modelo de chaves pré-publicadas no servidor. Ao instalar o aplicativo, o cliente gera pares de chaves baseados na curva elíptica \textit{Curve25519} e envia as partes públicas para o servidor \cite{antunes2018signal}:

\begin{itemize}
    \item \textbf{Identity Key ($IK_B$):} Uma chave de longo prazo que identifica a conta de Bob (vinculada ao usuário).
    \item \textbf{Signed Pre-Key ($SPK_B$):} Uma chave de médio prazo, assinada pela $IK_B$ para garantir autenticidade, renovada periodicamente.
    \item \textbf{One-Time Pre-Keys ($OPK_B$):} Um lote de chaves efêmeras de uso único. O servidor entrega uma dessas chaves para cada nova solicitação de início de conversa e a remove do banco de dados imediatamente, garantindo que chaves antigas não possam ser reutilizadas.
\end{itemize}

Nesta arquitetura, o servidor atua como um diretório de distribuição de chaves (\textit{Key Store}), sem nunca ter acesso às chaves privadas correspondentes, que permanecem exclusivamente nos dispositivos dos usuários.

\subsection{Estabelecimento de Sessão: X3DH}
O processo de "aperto de mão" (\textit{handshake}) inicial é denominado \textbf{X3DH} (\textit{Extended Triple Diffie-Hellman}). Quando Alice deseja iniciar uma conversa, ela obtém o pacote de pré-chaves de Bob no servidor e realiza cálculos locais para derivar um segredo compartilhado ($SK$).

O X3DH combina chaves de longo prazo e efêmeras para garantir autenticação mútua e sigilo encaminhado desde a primeira mensagem \cite{cohn2017formal}. O segredo $SK$ é calculado através da concatenação dos resultados de operações ECDH:

\begin{equation}
    SK = KDF(DH1 \ || \ DH2 \ || \ DH3 \ || \ DH4)
\end{equation}

Onde os componentes são definidos como:
\begin{itemize}
    \item $DH1 = DH(IK_A, SPK_B)$: Autenticação mútua e identificação baseada nas chaves de longo prazo e assinadas.
    \item $DH2 = DH(E_A, IK_B)$: Autenticação da chave efêmera de Alice contra a identidade de Bob.
    \item $DH3 = DH(E_A, SPK_B)$: Combinação de chaves efêmeras e assinadas para fortalecer a sessão.
    \item $DH4 = DH(E_A, OPK_B)$: Garante o sigilo perfeito encaminhado inicial, pois a $OPK_B$ é descartada após o uso.
\end{itemize}

Após o cálculo, Alice apaga sua chave efêmera privada ($E_A$) e usa o $SK$ para inicializar o algoritmo \textit{Double Ratchet}. Ela envia sua chave pública efêmera junto com a primeira mensagem cifrada para que Bob possa reproduzir o cálculo e derivar o mesmo segredo.

\subsection{O Algoritmo Double Ratchet}
Uma vez estabelecida a sessão via X3DH, o protocolo utiliza o algoritmo \textbf{Double Ratchet} para atualizar as chaves a cada mensagem trocada. Conforme documentado nas especificações oficiais \cite{perrin2016double}, o nome "Catraca" (\textit{Ratchet}) refere-se a um mecanismo que permite o avanço do estado criptográfico, mas impede matematicamente sua reversão. O algoritmo combina dois tipos de catracas:

\subsubsection{Ratchet Simétrico (Cadeia KDF)}
Esta catraca avança a cada mensagem enviada ou recebida, garantindo que cada pacote de dados utilize uma chave única. O protocolo emprega uma Função de Derivação de Chaves (KDF) baseada em HMAC-SHA256. A partir de uma \textit{Chain Key} ($CK_{i}$), o algoritmo deriva duas saídas:
\begin{enumerate}
    \item Uma \textbf{Message Key} ($MK_{i}$), usada exclusivamente para cifrar o conteúdo da mensagem atual.
    \item A próxima \textbf{Chain Key} ($CK_{i+1}$), que será usada na iteração seguinte.
\end{enumerate}

\begin{equation}
    MK_i = KDF_{mk}(CK_i) \quad \text{e} \quad CK_{i+1} = KDF_{ck}(CK_i)
\end{equation}

Como a KDF é uma função unidirecional, mesmo que um atacante obtenha a chave $MK_i$, ele não consegue calcular as chaves anteriores ($MK_{i-1}$), garantindo o \textbf{Sigilo Perfeito Encaminhado}.

\subsubsection{Ratchet Diffie-Hellman (Auto-Cura)}
O Ratchet Simétrico protege as mensagens passadas, mas é vulnerável se a \textit{Chain Key} atual for comprometida. Para mitigar isso, o protocolo executa um novo ECDH sempre que a conversa muda de direção (ex: Bob responde Alice), atualizando a entropia do sistema.

As mensagens carregam novas chaves públicas efêmeras. Quando Bob responde, ele envia um novo valor DH público. Alice combina essa chave com sua chave privada para derivar uma nova "Raiz" de criptografia ($Root Key$). Isso reinicia as cadeias simétricas com material criptográfico novo, garantindo a \textbf{Segurança Pós-Comprometimento} (ou auto-cura) \cite{perrin2016double}.

A Figura~\ref{fig:double_ratchet} ilustra a interação entre a Cadeia Raiz (gerenciada pelo DH Ratchet) e as Cadeias de Envio/Recepção (gerenciadas pelo Ratchet Simétrico).

\begin{figure}[ht]
    \centering
    \begin{tikzpicture}[
        node distance=1.5cm and 2cm,
        keynode/.style={circle, draw=black, thick, minimum size=1cm, align=center, font=\small},
        msgnode/.style={rectangle, draw=black, fill=gray!10, minimum size=0.8cm, rounded corners},
        arrow/.style={-Latex, thick}
    ]

    % Root Key Chain
    \node[keynode, fill=blue!10] (RK1) {RK$_{1}$};
    \node[keynode, fill=blue!10, right=of RK1] (RK2) {RK$_{2}$};
    
    % Sending Chain (Symmetric)
    \node[keynode, fill=green!10, below=1cm of RK1] (CK1) {CK$_{1,0}$};
    \node[keynode, fill=green!10, below=1cm of CK1] (CK2) {CK$_{1,1}$};
    
    % Message Keys
    \node[msgnode, right=1cm of CK1] (MK1) {MK$_{0}$};
    \node[msgnode, right=1cm of CK2] (MK2) {MK$_{1}$};

    % Arrows Root Chain
    \draw[arrow] (RK1) -- node[above] {\scriptsize DH Ratchet} (RK2);
    \draw[arrow] (RK1) -- node[left] {\scriptsize KDF} (CK1);
    
    % Arrows Symmetric Chain
    \draw[arrow] (CK1) -- node[above] {\scriptsize Output} (MK1);
    \draw[arrow] (CK1) -- node[left] {\scriptsize KDF} (CK2);
    \draw[arrow] (CK2) -- node[above] {\scriptsize Output} (MK2);
    \draw[arrow, dashed] (CK2) -- +(0,-1) node[below] {\scriptsize Próxima CK};

    % Labels
    \node[above=0.2cm of RK1, font=\bfseries] {Cadeia Raiz (Diffie-Hellman)};
    \node[left=0.2cm of CK1, font=\bfseries, rotate=90, anchor=south] {Cadeia de Envio (Simétrica)};

    \end{tikzpicture}
    \caption{Representação do Double Ratchet. A Cadeia Raiz (RK) é atualizada via Diffie-Hellman e alimenta a Cadeia de Envio (CK). A Cadeia de Envio gera chaves de mensagem (MK) individuais via KDF simétrico.}
    \label{fig:double_ratchet}
\end{figure}

Em suma, o Protocolo Signal utiliza o X3DH para o acordo inicial seguro e o Double Ratchet para manter a sessão protegida indefinidamente, "cicatrizando" a segurança da conversa automaticamente após qualquer comprometimento temporário.

% SEÇÃO 4: ESTUDO DE CASO - WESLEY 
\section{Estudos de Caso e Implementações}

\subsection{WhatsApp: O Paradoxo do Código Fechado e a Implementação do Signal}
O WhatsApp é a implementação mais popular do Protocolo Signal. Embora o protocolo original seja licenciado sob a GPLv3 (que exigiria que softwares derivados também fossem de código aberto), o WhatsApp mantém seu código fechado de forma proprietária. Isso foi viabilizado por um acordo comercial firmado em 2014 com a Open Whisper Systems, permitindo a integração da biblioteca \textit{libsignal} em sua infraestrutura antes que as obrigações estritas da GPL fossem aplicadas à sua base de código final \cite{marlinspike2016whatsapp}.

O fato de o aplicativo ser de código fechado (\textit{closed-source}) cria um paradoxo técnico. Enquanto o Protocolo Signal é de código aberto e possui sua matemática amplamente auditada pela comunidade de segurança, a sua implementação no WhatsApp exige confiança na Meta de que o aplicativo não possui "portas dos fundos" (\textit{backdoors}) projetadas para copiar mensagens antes da etapa de criptografia. Embora auditorias independentes atestem a integridade da criptografia de conteúdo da plataforma, essa arquitetura fechada serve diretamente ao modelo de negócios da empresa.

\subsubsection{A Engenharia de Implementação do Signal no WhatsApp}
Para atender a bilhões de usuários sem comprometer a eficiência dos dispositivos móveis ou consumir excessivamente a largura de banda, o WhatsApp realiza adaptações estruturais no uso do Signal. A arquitetura técnica resolve problemas práticos de comunicação assíncrona, transferência de arquivos pesados e escalabilidade em grupos.

Devido à natureza de código fechado do aplicativo, a compreensão exata de como essas adaptações funcionam requer técnicas de engenharia reversa. Conforme demonstrado por Antunes e Kowada \cite{antunes2018signal}, que utilizaram injeção dinâmica de código (via ferramenta FRIDA) para interceptar e depurar a comunicação em dispositivos Android, o WhatsApp implementa a matemática do Signal de forma íntegra. O estudo empírico revelou que as chaves e os textos cifrados são empacotados em estruturas de classes específicas (como \textit{Signal Message} e \textit{Pre Key Signal Message}) e transmitidos aos servidores por meio do \textit{FunXMPP}, uma variação altamente otimizada do protocolo XMPP que utiliza dicionários de bytes para comprimir requisições e economizar dados móveis \cite{antunes2018signal}. 

A partir dessas constatações práticas, é possível detalhar o funcionamento dos seguintes mecanismos:

\paragraph{O Repositório de Chaves (Servidor)}
O servidor do WhatsApp atua como um intermediário na troca de chaves, comparável a um "cartório digital". Para o funcionamento do protocolo X3DH (necessário para o envio de mensagens a usuários \textit{offline}), o servidor armazena as chaves públicas dos usuários: a Chave de Identidade, uma Chave Pré-assinada e lotes de Chaves de Uso Único. Ao iniciar uma conversa, o servidor entrega ao remetente uma Chave de Uso Único do destinatário e a descarta. O servidor facilita o encontro das chaves públicas, mas não possui acesso às chaves privadas, que permanecem restritas aos dispositivos.

\paragraph{Envio de Mídia e Arquivos}
O algoritmo \textit{Double Ratchet} não é otimizado para criptografar arquivos grandes diretamente. Para contornar essa limitação, o WhatsApp adota uma abordagem híbrida:
\begin{enumerate}
    \item \textbf{Criptografia Simétrica:} O aplicativo gera uma chave simétrica temporária e aleatória (padrão AES-256) e criptografa o arquivo de mídia.
    \item \textbf{Armazenamento em Nuvem:} O arquivo cifrado é enviado para os servidores da Meta.
    \item \textbf{Entrega Segura:} A chave temporária que destranca o arquivo, junto com seu link de acesso, é transmitida ao destinatário usando o canal criptografado de ponta-a-ponta do Protocolo Signal.
    \item \textbf{Acesso:} O destinatário recebe a mensagem segura, baixa o arquivo criptografado do servidor e utiliza a chave temporária para revelá-lo localmente.
\end{enumerate}

\paragraph{O Desafio da Escalabilidade em Grupos (\textit{Sender Keys})}
Para evitar a ineficiência de criptografar e transmitir a mesma mensagem individualmente para centenas de participantes de um grupo, a implementação utiliza o componente \textit{Sender Keys} (Chaves de Remetente). Quando um usuário envia sua primeira mensagem em um grupo, ele gera uma \textit{Sender Key} e a distribui individualmente para os demais membros utilizando sessões seguras par-a-par padrão do Signal. A partir de então, as mensagens destinadas ao grupo são criptografadas apenas uma vez utilizando essa chave, e o servidor se encarrega de distribuir a mensagem cifrada. Caso um membro saia do grupo, os participantes remanescentes rotacionam suas chaves para impedir acessos futuros pelo ex-membro.

\paragraph{Chamadas de Voz e Vídeo (SRTP)}
Para a comunicação em tempo real, o Protocolo Signal é utilizado apenas na fase de estabelecimento da chamada, negociando de forma segura uma chave secreta entre os dispositivos. Após o atendimento, a transmissão contínua de mídia é delegada ao protocolo SRTP (\textit{Secure Real-time Transport Protocol}), que utiliza a chave acordada para criptografar o fluxo de voz e vídeo com baixa latência.

\subsubsection{O Tesouro da Meta: A Coleta de Metadados}
Apesar de o conteúdo das mensagens no WhatsApp ser rigorosamente protegido pela criptografia supracitada, o modelo de negócios da Meta baseia-se fortemente na exploração de metadados. Em uma analogia simples, enquanto a criptografia protege a "carta" dentro do envelope, os metadados representam todas as informações escritas do lado de fora.

Ao contrário do aplicativo nativo do Signal, que desenhou seu sistema para minimizar o conhecimento do servidor através de técnicas como \textit{Sealed Sender}, o WhatsApp centraliza os metadados. A plataforma rastreia com precisão com quem o usuário se comunica, a frequência das interações, os horários de atividade, a localização aproximada e os dados telemétricos do aparelho. Ao cruzar esses dados com os bancos do Facebook e do Instagram, a Meta consegue traçar perfis comportamentais detalhados e valiosos para o direcionamento de anúncios.

\subsubsection{A Polêmica Atualização de 2021: O Antes e o Depois}
Em janeiro de 2021, o WhatsApp atualizou seus Termos de Serviço e Política de Privacidade, gerando uma reação global que resultou em uma migração em massa para aplicativos concorrentes. A polêmica foi cercada de desinformação: o WhatsApp não passaria a ler as mensagens, mas sim tornaria explícita a coleta e a integração de dados não protegidos pela criptografia \cite{whatsapp_privacy_2021}.

A mudança representou um marco nas políticas de compartilhamento:
\begin{itemize}
    \item \textbf{Antes de 2021:} Quando a integração com o Facebook começou a se intensificar, os usuários tiveram uma janela de tempo para realizar o \textit{opt-out}, recusando o uso de seus dados para anúncios.
    \item \textbf{A partir de 2021:} A aceitação da nova política tornou-se obrigatória. O foco principal foram as interações com Contas Comerciais (WhatsApp Business). A política estabeleceu que, ao interagir com empresas que utilizam os serviços de hospedagem da Meta, a própria plataforma teria acesso aos metadados dessas interações para gerenciar mensagens e direcionar anúncios personalizados no ecossistema da empresa.
\end{itemize}

Conforme detalhado na seção "Informações que coletamos" da política de privacidade atualizada, os metadados retidos incluem dados de atividade, grafo social (quem interage com quem), informações do dispositivo e endereço IP.

\subsubsection{Exemplo Prático: Resposta a Ordens Judiciais}
A distinção entre o sigilo do conteúdo e a exposição do comportamento torna-se evidente em cenários de requisições judiciais. Caso um juiz determine que a Meta entregue os dados de um usuário investigado, a resposta técnica varia drasticamente:

O texto das mensagens, áudios, fotos e vídeos nunca pôde ser entregue, pois os servidores armazenam apenas códigos ilegíveis. Antes das integrações mais profundas da Meta, as informações fornecidas à justiça limitavam-se a dados cadastrais básicos (número de telefone, \textit{last seen} e endereço IP). 

No cenário atual, a riqueza dos metadados permite a entrega de informações táticas fundamentais. A Meta pode fornecer a teia de contatos do investigado (frequência e registros de mensagens), dados de grupos (informações de todos os membros associados) e vínculos sociais com perfis do Facebook e Instagram. Consequentemente, mesmo sem acesso ao teor das conversas, as autoridades conseguem mapear de forma cristalina a rede de relacionamentos de uma investigação.

\subsection{Matrix e Olm/Megolm}
\textcolor{red}{[LA - WESLEY]}
\textcolor{red}{[aqui é bom fazer uma menção honrosa para essa implementação promissora do signal pela Matrix...]}
O Matrix é um padrão aberto para comunicação descentralizada e federada. Diferente do Signal (que é centralizado), o Matrix permite que servidores diferentes conversem entre si.

Para garantir segurança nesse ambiente, o Matrix desenvolveu a biblioteca \textbf{Olm}. Olm é uma implementação do algoritmo Double Ratchet do Signal, escrita em C++ e adaptada para a arquitetura do Matrix.

No entanto, o algoritmo original do Signal é "pesado" para grupos grandes (complexidade $O(N^2)$), pois exige o estabelecimento de sessões individuais par-a-par. Para resolver isso, o Matrix introduziu o \textbf{Megolm}. O Megolm sacrifica a propriedade de "auto-cura" imediata em troca de escalabilidade massiva para grupos com milhares de usuários.

No Megolm, cada participante cria uma sessão de saída ("outbound") e compartilha a chave da catraca com os outros participantes via canais seguros Olm \cite{matrix_spec}. Isso permite criptografia eficiente em ambientes federados sem quebrar as propriedades fundamentais de confidencialidade.

% SEÇÃO 5: CONCLUSÃO - TODOS 

\section{Conclusão}
\textcolor{red}{[TODOS][]}
A evolução do algoritmo Axolotl para o onipresente Protocolo Signal redefiniu a segurança na internet. A tecnologia provou ser robusta contra adversários estatais e corporativos no que tange à confidencialidade do conteúdo.

Entretanto, a análise das implementações do WhatsApp e Matrix revela que o protocolo é apenas uma peça do quebra-cabeça. No caso do WhatsApp, a segurança criptográfica convive com a vigilância de metadados sancionada pelos termos de uso. No caso do Matrix, vemos a adaptação do protocolo (Megolm) para garantir soberania de dados através da federação. Conclui-se que a "privacidade" é um conceito mais amplo que a "criptografia", dependendo tanto da matemática dos algoritmos quanto da ética das implementações.










\section{Estudos de Caso e Implementações}

\subsection{WhatsApp: Privacidade vs. Metadados}
O WhatsApp é a implementação mais popular do Protocolo Signal. Embora o protocolo seja licenciado sob a GPLv3 (que exigiria que softwares derivados também fossem de código aberto), o WhatsApp mantém seu código fechado (proprietário).

Isso é possível porque o WhatsApp não simplesmente copiou o código do repositório público; eles firmaram um acordo comercial específico com a Open Whisper Systems em 2014 para integrar a biblioteca \textit{libsignal} em sua infraestrutura antes que as obrigações estritas da GPL fossem aplicadas à sua base de código final \cite{marlinspike2016whatsapp}.

\subsubsection{A Polêmica dos Metadados (2021)}
Embora o conteúdo das mensagens no WhatsApp seja protegido pela mesma criptografia do Signal, o modelo de negócios da Meta (controladora do WhatsApp) baseia-se na exploração de \textbf{metadados}.

Em janeiro de 2021, o WhatsApp atualizou seus Termos de Serviço e Política de Privacidade, gerando reação global e migração em massa para o Signal e Telegram. A atualização explicitou a coleta de dados que não são protegidos pela criptografia ponta-a-ponta. Conforme a seção "Informações que coletamos" da política \cite{whatsapp_privacy_2021}:

\begin{itemize}
    \item \textbf{Dados de Atividade:} Tempo, frequência e duração das interações.
    \item \textbf{Grafo Social:} Quem fala com quem (remetente e destinatário).
    \item \textbf{Informações do Dispositivo:} Modelo de hardware, sistema operacional, nível de bateria, força do sinal e identificadores únicos.
    \item \textbf{Endereço IP:} Usado para estimar a localização geral.
\end{itemize}

Ao contrário do aplicativo Signal, que desenhou seu sistema para minimizar o conhecimento do servidor (usando técnicas como \textit{Sealed Sender}), o WhatsApp centraliza metadados, permitindo a construção de perfis comportamentais detalhados dos usuários, mesmo sem ler o conteúdo textual das mensagens.

\subsection{Matrix e Olm/Megolm}
O Matrix é um padrão aberto para comunicação descentralizada e federada. Diferente do Signal (que é centralizado), o Matrix permite que servidores diferentes conversem entre si. Para garantir segurança nesse ambiente, o Matrix desenvolveu a biblioteca \textbf{Olm}.

Olm é uma implementação do algoritmo Double Ratchet do Signal, escrita em C++ e adaptada para a arquitetura do Matrix. No entanto, o algoritmo original do Signal é "pesado" para grupos grandes (complexidade $O(N^2)$), pois exige o estabelecimento de sessões individuais par-a-par.

Para resolver isso, o Matrix introduziu o \textbf{Megolm}. O Megolm sacrifica a propriedade de "auto-cura" imediata (o \textit{break-in recovery} é mais lento) em troca de escalabilidade massiva para grupos com milhares de usuários. No Megolm, cada participante cria uma sessão de saída ("outbound") e compartilha a chave da catraca com os outros participantes via canais seguros Olm \cite{matrix_spec}. Isso permite criptografia eficiente em ambientes federados sem quebrar as propriedades fundamentais de confidencialidade.

\section{Conclusão}
A evolução do algoritmo Axolotl para o onipresente Protocolo Signal redefiniu a segurança na internet. A tecnologia provou ser robusta contra adversários estatais e corporativos no que tange à confidencialidade do conteúdo.

Entretanto, a análise das implementações do WhatsApp e Matrix revela que o protocolo é apenas uma peça do quebra-cabeça. No caso do WhatsApp, a segurança criptográfica convive com a vigilância de metadados sancionada pelos termos de uso. No caso do Matrix, vemos a adaptação do protocolo (Megolm) para garantir soberania de dados através da federação. Conclui-se que a "privacidade" é um conceito mais amplo que a "criptografia", dependendo tanto da matemática dos algoritmos quanto da ética das implementações.

\bibliographystyle{bib/sbc}
\bibliography{bib/sbc-template}

\end{document}